\documentclass[11pt, a4paper]{article}

% ====== Preamble ======
% 前導區:統一載入樣式與文獻設定
\usepackage{wmt-template}
\addbibresource{references.bib}
\graphicspath{{./}{./figures/}}

\algnewcommand{\To}{\textbf{to }}
\algnewcommand{\Step}{\textbf{step }}

\studentname{Your  Name}
\title{Your Paper Title}
\author{}
\date{}

\begin{document}

\maketitle
\thispagestyle{fancy}

\begin{abstract}
% Keep abstract here, or move to a separate file with:\input{chapters/00-abstract}
% 中文提示:摘要請維持 200--250 字,並回答 Why/What/How/So What。
This paper addresses a core challenge in modern data-intensive systems and proposes a practical, high-performance solution. We identify key limitations of existing methods, including scalability bottlenecks and unstable efficiency under large workloads. To solve this problem, we design a method that combines structured data processing, optimized computation flow, and robust filtering mechanisms. The proposed framework is implemented and evaluated on representative datasets under controlled settings. Results show consistent improvements in runtime efficiency, memory usage, and overall robustness compared with strong baselines. These findings indicate that the method is suitable for large-scale deployment scenarios and provides measurable value for both research and real-world system operations.
\end{abstract}

\section{Introduction}
% 中文提示:章節內容在 chapters/01-introduction.tex
% !TeX root = ../manuscript.tex
% Chapter content placeholder: Introduction
% Keep funnel logic: background -> gap -> approach -> contributions.
Data-intensive systems increasingly require mining pipelines that are both efficient and robust under variable workloads. Existing methods often optimize one objective at the cost of another, which limits deployment reliability. This motivates a framework that balances runtime efficiency, memory control, and analytical consistency.

As shown in recent system-oriented studies \cite{Wu2024Systems} and \cite{Wu2025ICDM}, balancing throughput and quality remains a central challenge in real deployments.

However, three practical gaps remain. First, many pipelines assume stable data distributions and degrade when workload skew changes rapidly. Second, memory control is often treated as an afterthought, leading to unstable peak usage in production traffic windows. Third, evaluation protocols are frequently difficult to reproduce because setup details and baseline configurations are under-specified.

To address these gaps, this reference workflow uses a chapter-structured method description with explicit constraints on problem definition, algorithm presentation, and experimental reporting. The objective is not only to present a method, but also to make the drafting process auditable and easier to iterate under advisor or reviewer feedback.


\section{Background and Preliminaries}
% 中文提示:Related Work + Preliminaries 需對齊規範
% !TeX root = ../manuscript.tex
% Chapter content placeholder: Background and Preliminaries
% Include categorized related work and formal Given/Find/Such that statement.


\section{Proposed Algorithm}
% 中文提示:方法章需含 pseudocode 與複雜度分析
% !TeX root = ../manuscript.tex
% Chapter content placeholder: Proposed Algorithm
% Include overview, core strategy, pseudocode, and complexity analysis.


\section{An Illustrated Example}
% 中文提示:建議用 簡化範例 (simplified example) 展示逐步狀態變化
% !TeX root = ../manuscript.tex
% Chapter content placeholder: An Illustrated Example
% Provide toy data and step-by-step intermediate states.


\section{Experimental Results}
% 中文提示:實驗需交代 setup/datasets/metrics/baselines
% !TeX root = ../manuscript.tex
% Chapter content placeholder: Experimental Results
% Include setup, datasets, metrics, baselines, and causal analysis.


\section{Conclusion and Future Works}
% 中文提示:請明確寫出限制與未來方向
% !TeX root = ../manuscript.tex
% Chapter content placeholder: Conclusion and Future Works
% Summarize contributions, limitations, and realistic future directions.


\printbibliography

\end{document}
