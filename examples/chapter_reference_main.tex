% Chapter-based reference example main file.
% 中文說明:這份檔案是「章節式示例參考」,主要寫作請使用 manuscript.tex。
% 中文說明:想看各章如何寫,請看 examples/chapters/*.tex。

\documentclass[11pt, a4paper]{article}

% ====== Preamble ======
\makeatletter
\def\input@path{{./}{../}{examples/}}
\makeatother
\usepackage{wmt-template}

\IfFileExists{references.bib}{%
  \addbibresource{references.bib}%
}{%
  \addbibresource{../references.bib}%
}

\graphicspath{{./}{./figures/}}

\algnewcommand{\To}{\textbf{to }}
\algnewcommand{\Step}{\textbf{step }}

\studentname{Student Name}
\title{Chapter-Based Reference Template for Thesis Drafting}
\author{}
\date{}

\begin{document}

\maketitle
\thispagestyle{fancy}

\begin{abstract}
This reference template demonstrates a chapter-oriented writing workflow for academic drafting. It organizes the manuscript into modular sections and maps each section to a dedicated chapter example file. The design helps authors write incrementally while preserving consistent formatting, citation style, and cross-reference conventions. To keep quality and compliance visible, each chapter example follows explicit structural constraints, including contribution clarity in the introduction, formal notation in preliminaries, reproducible setup in experiments, and limitation disclosure in the conclusion. Authors should use this file as a learning reference and write their actual content in \texttt{manuscript.tex} with \texttt{chapters/*.tex}.
\end{abstract}

% ====== Chapter-oriented reference ======
% 中文說明:以下每章都對應到 examples/chapters 的示例檔。

\section{Introduction}
% !TeX root = ../manuscript.tex
% Chapter content placeholder: Introduction
% Keep funnel logic: background -> gap -> approach -> contributions.
Data-intensive systems increasingly require mining pipelines that are both efficient and robust under variable workloads. Existing methods often optimize one objective at the cost of another, which limits deployment reliability. This motivates a framework that balances runtime efficiency, memory control, and analytical consistency.

As shown in recent system-oriented studies \cite{Wu2024Systems} and \cite{Wu2025ICDM}, balancing throughput and quality remains a central challenge in real deployments.

However, three practical gaps remain. First, many pipelines assume stable data distributions and degrade when workload skew changes rapidly. Second, memory control is often treated as an afterthought, leading to unstable peak usage in production traffic windows. Third, evaluation protocols are frequently difficult to reproduce because setup details and baseline configurations are under-specified.

To address these gaps, this reference workflow uses a chapter-structured method description with explicit constraints on problem definition, algorithm presentation, and experimental reporting. The objective is not only to present a method, but also to make the drafting process auditable and easier to iterate under advisor or reviewer feedback.


\section{Background and Preliminaries}
% !TeX root = ../manuscript.tex
% Chapter content placeholder: Background and Preliminaries
% Include categorized related work and formal Given/Find/Such that statement.


\section{Proposed Algorithm}
% !TeX root = ../manuscript.tex
% Chapter content placeholder: Proposed Algorithm
% Include overview, core strategy, pseudocode, and complexity analysis.


\section{An Illustrated Example}
% !TeX root = ../manuscript.tex
% Chapter content placeholder: An Illustrated Example
% Provide toy data and step-by-step intermediate states.


\section{Experimental Results}
% !TeX root = ../manuscript.tex
% Chapter content placeholder: Experimental Results
% Include setup, datasets, metrics, baselines, and causal analysis.


\section{Conclusion and Future Works}
% !TeX root = ../manuscript.tex
% Chapter content placeholder: Conclusion and Future Works
% Summarize contributions, limitations, and realistic future directions.


\printbibliography

\end{document}
