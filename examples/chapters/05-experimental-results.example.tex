% !TeX root = ../chapter_reference_main.tex
% Experimental results example

\subsection*{Setup and Metrics}
Report CPU, RAM, OS, and implementation toolchain. Use runtime, peak memory, and scalability as primary metrics.

For reproducibility, always include dataset sizes, sparsity characteristics, and baseline versions. If hardware differs across runs, explicitly report the difference and avoid direct speedup claims across non-equivalent setups.

\subsection*{Baselines and Analysis}
Compare against strong baselines under identical settings. Beyond reporting numerical differences, explain why trends occur (e.g., pruning efficiency, reduced synchronization overhead, or cache behavior).

In this example chapter, numerical values are illustrative only. Use your real measurements in the final manuscript and keep interpretation causal: explain whether improvements come from reduced candidate volume, better memory locality, fewer synchronization points, or lower I/O overhead.
