% !TeX root = ../chapter_reference_main.tex
% Background and preliminaries example

\subsection*{Related Work (Categorized)}
Prior studies can be grouped into index-based, compression-based, and parallel scheduling approaches. Index-based methods reduce lookup overhead but may suffer from high maintenance cost. Compression-based methods lower memory usage yet can introduce additional decode latency. Parallel scheduling methods improve throughput but may become unstable under skewed task distributions.

For instance, Smith et al.~\cite{Wu2026TIFS} discuss privacy-preserving constraints that further complicate efficiency optimization in practical settings.

From a systems perspective, these categories expose a recurring tradeoff: methods that aggressively optimize throughput often increase implementation and tuning complexity, while methods that emphasize stability may sacrifice peak performance under bursty traffic. This tradeoff motivates a unified formulation that makes resource constraints explicit.

\subsection*{Problem Statement}
\textbf{Given} a dataset $D$, resource budget $B$, and quality target $\tau$.\\
\textbf{Find} a mining process $\Pi$ that generates result set $R$.\\
\textbf{Such that} runtime and memory remain within budget while meeting the target quality.

\subsection*{Notation}
Let $|D|$ denote the number of records, $k$ the candidate size per iteration, and $M_{peak}$ the peak memory footprint during execution. The optimization goal is to minimize execution cost while preserving result consistency:
\begin{equation}
\min\ C(\Pi) = \alpha T(\Pi) + \beta M_{peak}(\Pi), \quad \text{s.t. } Q(\Pi) \geq \tau
\label{eq:example_objective}
\end{equation}
